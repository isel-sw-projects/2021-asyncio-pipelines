% 
%  chapter2.tex
%  ThesisISEL
%  
%  Created by Matilde Pós-de-Mina Pato on 2012/10/09.
%
\chapter{State of the Art}
\label{cha:users_manual}

Firstly, on section \ref{sec:related_work}, will be made an overview on previously developted work made on this subject, then, on section \ref{sec:async_concepts}, will be made a characterization of the key concepts related to it. By last, on section \ref{sec:state_of_the_art}, are presented and explained several technologies representative of the state of the Art on asynchronous data flow in different programming realities, e.g. on Kotlin, JAVA and C\#. 


% ================
% = Related work =
% ================
\section{Background} % (fold)
\label{sec:related_work}

% Initially, when computers had a single processing unit, the networks only allowed the exchange of few bytes per second and the data servers didn't had the responsiveness requirements that are mandatory today; software computational systems were simpler, in the sense that operations were mostly made in a single execution thread. 
% The servers design, were made without many concerns about availability under demand pressure or computer resource optimization. 
% Consequently, operations that required intensive IO interactions or data request from an external sources, were mostly blocking, non-flexible in terms of responsiveness and interoperability. 

From the end of 80´s to the beginning of the 2000´s, with the acceleration of Moores´s Law in hardware and network bandwidth development, the creation of the web as we know today through the wide spread of use of the HTTP protocol and the support from new operative systems to multithreading support, the necessity of high responsiveness servers started to grow. This increase in demand of new ways to handle data through parallelism, caused the necessity of design new programming models compatible with concurrent work. 

Taking the wave initiated by the Gang of Four in \cite{gof}, where 23 patterns were compiled to deal with object-oriented problems, a group of researchers published the \textit{Proactor Pattern} in the paper \cite{proactor} to deal with asynchronous IO. 
In the document, are identified four properties that high-performance web server must have: 
\begin{itemize}
	\item Concurrency - The server must process multiple client requests simultaneously.\\
	\item Efficiency - The software design must be built aiming the use of least hardware resources as possible. \\
	\item Simplicity - The code of the solution must be easy to understand, modular and avoid own built design patterns as possible. \\
    \item Adaptability - The system must be totally decoupled from client implementations, allowing it to be easily used by any client independently of the underlying technologic realities. To achieve this, may be used standards e.g. \cite{REST} or SOAP.\\
\end{itemize}

The authors propose the \textit{Proactor Pattern}, because in their opinion, conventional concurrency models fail to fully achieve the enumerated properties. In the paper, before presenting the \textit{Proactor Pattern}, are identified two major concurrency models, namely: \textit{multithreading} and \textit{reactive event dispatching}. 

The paper refers that one of the most direct implementations of the multithreading approach, is the handling of multiple requests by creating a new thread every request. Each request will then be fully processed and the recently  created thread is then be disposed after the work is finished. 

This solution has several serious issues. Firstly, creating a new thread per request is highly costly in terms of computational resources,
because are involved context switches between user and kernel modes; secondly, must be taken in account synchronization to maintain data integrity.
Then, the authors warn about the fact that the IO retrieved data is mainly memory-mappped, wich rises the question: What happens when the data obtained through IO becomes greater than the system memory can hold? The system stalls until more memory becomes available!?
On last, if the server receives a high demand of requests, the server easily blocks in the process of creating and disposing threads. 

To avoid this issue, the authors, recommended the use of dynamic threadpools to process requests, where each request will be linked to a pre-existing thread, avoiding all the overhead of creating and disposing a thread per request;
however, issues related with memory-mapping and overhead due to the switching of data between different threads maintains. 

Another traditional concurrency model identified by the authors of the paper, is the \textit{Reactive Synchronous Event Dispatching} or more commonly known as \texttt{Reactor Pattern}. In this model, a \textit{Dispatcher}, with a single thread in a loop, is constantly listening requests from clients and sending work requests to an entity named \textit{Handler}. 
The \textit{Handler}, will then process the IO work Synchronously and request a connection to the client in the \textit{Dispatcher}. When the requested connection is ready to be used, the \textit{Dispatcher} notifies the \textit{Handler}. After the notification, the \textit{Handler} asynchronously sends the data, that is being or has been obtained through IO, to the client.\\
Although the authors identifying that this approach is positive, because decouples the application logic from the dispatching mechanisms besided with the low overhead due the use of a single thread, the authors identify several drawbacks with this approach. 
Firstly, since IO operation are synchronous, the code for this approach is complex because must be set in place mechanisms to avoid IO blocking through hand off mechanisms. 
Then, if a request processing blocks, the processing of another requests may be impacted. 

To keep the positive points but mitigating the identified issues of previous approaches, is suggested the \textit{Proactor Pattern}. 
This pattern is very similar to the \textit{Reactive Synchronous Event Dispatching}, however, after the requests processed by a single threaded \textit{Completion dispatcher}, 
the IO work is then dispatched asynchronously to the underlying OS IO subsystems, where multiple requests can be processed simultaneously. 
For the result to be retrieved, is previously registered a callback in the \textit{Completion Dispatcher} and the OS has the responsibility to queue the finished result in a well known place. 

Finally, the \textit{Completion Dispatcher} has the responsibility to dequeue the result placed by the OS and call the correct previously registered callback. 
With this, this model creates a platform that provides: decoupling between application and processing mechanisms,
offers concurrent work without the issues inherent with the use of threading mechanisms and since IO is managed by the OS subsystems, is avoided code complexity in handling possible blocking and scheduling issues.  

The \textit{Proactor Pattern}, creates the ground for several models used by modern platforms that use a single/few threads to process client requests and parallel mechanisms to do the heavy work in the background; namely, for example: \textit{Javascript NODE.JS}, \textit{Spring Webflux}, \textit{vertx} and others.

From what was explained until now, is evident the tendency followed by software architects in terms of asynchronous processing from non-reactive to event driven approaches. Initially the systems were non-reactive, where each request had to be processed in a 
specific thread and that thread blocked until something got ready to go further. 
Then, with the asynchronous systems based on events with the introduction of callback systems inspired in patterns like the \textit{Reactor} or \textit{Proactor}; the software design started to become more event driven, allowing the servers to be more efficient in responsiveness, flexibility and resources optimization. 

However, are some limitations in these asynchronous models. For example, if the data to be processed is bigger than the memory available or if the data to be calculated is from a source that produces data at a constant rate that must be processed in real time, these models work badly.  
The traditional models fail to comply these objectives because are mostly eager by design or not comply with the notion of a continuous source of information that requires to be processed in real time.
Taken this in account, projects like project Reactor, Asynchronous Enumerable provided by Microsoft or papers like \cite{LAZYVSEAGER}, try to deal with these issues, by providing API's that merge the concepts of Fluent API´s, functional programming and code syntax that tries to resemble synchronous code, being the complexity inherent with asynchronous models implementations hidden from the programmer. 

%\begin{figure}[ht]
%	\centering
%	\includegraphics[width=0.65\textwidth]{alternative_1}
%	  \caption{Multithreading solution example}
 % \label{fig:bibtex}
%\end{figure}


%\begin{figure}[ht]
%	\centering
	%\includegraphics[width=0.65\textwidth]{alternative_2}
%	  \caption{Dispatcher example}
 % \label{fig:bibtex}
%\end{figure}

%\begin{figure}[ht]
%	\centering
%	\includegraphics[width=0.65\textwidth]{alternative_3}
%	  \caption{Proactor example}
%  \label{fig:bibtex}
%\end{figure} %

% section introduction (end)

% ====================
% = Folder Structure =
% ====================
\section{Asynchronous flow key concepts and design alternatives} % (fold)
\label{sec:async_concepts}

With the development of several approaches and implementations related to asynchronous data flow in several programming plataforms; a dictionary
of properties, concepts and design alternatives started to grow by itself. In the following, are discussed several of the concepts related with asynchronous data flow, namely:


\subsection{Synchronous versus Asynchronous}

	Before explaining more terms related with asynchronous data flow, it's important to clarify what is synchrony and asynchronous in programming. 
	
	Asynchronous in programming, is a call to a function or routine that returns immediately, not blocking the caller until the operation is finished. The operation processing, will be completely independent from the caller execution process and can even be done in another machine. This way, the caller is freed to do more work, even to start \texttt{N} more operations in parallel. 
	
	Meanwhile, a call to a synchronous function or routine, blocks the caller until the operation finishes. In this case, the caller has to wait for the completion of the synchronous operation before going forward, which limits the program efficiency if parallelism is applicable.

	To better visualize what was explained, we have the following example: 


	\begin{figure1}[htbp]
		\centering
		\begin{subfigure}[h]{1.2\textwidth}
			\centering
			\lstfromfile{java}{1-14}{Asynchronous call example}{async}{showlines=true,morekeywords={begin,System,out,print},numbers=left, firstnumber=1}{async.java}
			\caption{}
			\label{fig:ra-vectorial}
		 \end{subfigure}	
	\qquad
		 \begin{subfigure}[h]{1.2\textwidth}
			\centering
			\lstfromfile{java}{1-15}{Synchronous call example}{Sync}{showlines=true,morekeywords={begin,System,out,print},numbers=left, firstnumber=1}{Sync.java}
			\caption{}
			\label{fig:ra-raster}
		\end{subfigure}		
	  \caption{Example of Synchronous and Asynchronous calls in JAVA}
	  \label{fig:figura-completa}
	\end{figure1}

	As we can see, in the synchronous call example, the operation return only happens after the whole subsequent remote operation is finished, consequently, the caller operation is dependent from several variables to go forward e.g. : HTTP messaging latency, remote server operation speed or bandwidth issues. 

	Meanwhile, in the asynchronous operation call, the return happens immediately after the call, however, the processing inherent with that operation will start just when the subsystem that handles the asynchronous function is ready to process that work, for example, when a worker the OS is ready to process the received messages from the remote HTTP server that handled the request.






	\subsection{Push vs Pull} 
	
	Another concept important to understand how asynchronous data flow is handled in programming, is the \textit{Pull} and \textit{Push} processing patterns.
	In \textit{Pull} pattern, usually, exists a source of data and the program iterates over that source to operate over each item. 
	
	On the other hand, in the \textit{Push} pattern, the items of the data source are "Pushed" to a routine that will operate over that item.
	To help to assimilate what was just explained, we have the following example:

	\begin{figure2}
		\centering
		\begin{subfigure}[h]{1.2\textwidth}
			\centering
			\lstfromfile{java}{1-17}{Pull pattern example}{pull}{showlines=true,morekeywords={begin,System,out,print},numbers=left, firstnumber=1}{pull.java}
			\caption{}
		 \end{subfigure}	
	\qquad\qquad
		 \begin{subfigure}[h]{1.2\textwidth}
			\centering
			\lstfromfile{java}{1-17}{Push pattern example}{push}{showlines=true,morekeywords={begin,System,out,print},numbers=left, firstnumber=1}{push.java}
			\caption{}
		\end{subfigure}		
	  \caption{Example of Pull and Push data handling patterns}
	  \label{fig:exmplo2}
	\end{figure2}
	
	As we can see, in the pull pattern, the items are "pulled" from a data source through an iteration mechanism. 
	
	In contrast with that, in the \textit{Push} pattern, items are pushed to a consumer through a supplier.





	\subsection{Reactive Streams} 

	\textit{Reactive Streams} is term used to describe several programming approaches, designed to asynchronously process data in real time obtained from data sources that continuously produce. 



	\subsection{Hot versus Cold}
	
	Another property that must be took in account when handling with \textit{Reactive Streams} or asynchronous data processing in general, is the nature of the data flow. There are two main adjectives to name a data flow, \texttt{Hot} or \texttt{Cold}.  \\

	A \texttt{Cold} data flow, is a flow of information that is produced just when the stream pipeline in subscribed by an observer. In this case, the producer only starts sending/producing data when someone is interested in the data from that source. 
	For example, when program uses a IO mechanism to lazily retrieve a sequence of words from a database, the IO mechanism will only start sending information just when a consumer subscribes that data flow. Usually, the data is sent to the consumer in unicast.

	On the other hand, in a \texttt{Hot} data flow, the data is produced independently of existing any observer to that information. This mechanism usually work in broadcast and the data is continuously produced and sent to possible observers.
	In this case, when a observer subscribes to a publisher, exists the possibility of data items being already lost to that publisher while in the \texttt{Cold} flow, the consumer usually receives all items that were produced by the source.
	In the following examples, a number is produced each 100 miliseconds:

	\begin{figure3}
		\centering
		\begin{subfigure}[h]{1.2\textwidth}
			\centering
			\lstfromfile{java}{1-3}{Cold Example}{cold}{showlines=true,morekeywords={begin,System,out,print},numbers=left, firstnumber=1}{ColdFlux.java}
			\caption{}
		 \end{subfigure}	
	\qquad\qquad
		 \begin{subfigure}[h]{1.2\textwidth}
			\centering
			\lstfromfile{java}{1-6}{Hot Example}{hot}{showlines=true,morekeywords={begin,System,out,print},numbers=left, firstnumber=1}{HotFlux.java}
			\caption{}
		\end{subfigure}		
	  \caption{Example of Hot and Cold data streams}
	  \label{fig:exmplo3}
	\end{figure3}

	
	As we can see, in the \texttt{Hot} data flow example, the items are emitted from the moment the producer is created, independently of existing any subscriber or observer attached to that publisher. Notice that when a consumer is subscribed to the publisher, 1 seconds after the emition started, the numbers from 0 to 10 were not printed.
	
	In the \texttt{Cold} example, the producer only emits data when a subscription is done. Being all the produced numbers printed because the emition started just when a subscription was made. 

	\subsection{Cancelables} todo\\
	\subsection{Error Handling}  todo\\
	\subsection{Intrisic Keywords} todo\\
    \subsection{Async Enumerables} todo\\ 

    

% The template file for writing dissertations in  \texttt{LaTeX} is organized into a main directory, a set of files and sub-directories:
% \begin{enumerate}
% 	\item[ThesisISEL] This is the main directory and includes:
% 	\begin{enumerate}
% 		\item \textbf{Logo} Directory with Faculty logos;
% 		\item \textbf{sty} Directory will all sty files that help in formatting document;
% 		\item \textbf{Chapters} Directory where to put user files (text and figures);
% 		\begin{enumerate}
% 		\item \textbf{scripts} Directory with useful bash scripts, e.g., for cleaning all temporary files;
% 		\item \textbf{img} Directory with all images of your thesis;
% 		\end{enumerate}
% 		\item \textbf{alpha-pt.bst} A file with bibliography names in portuguese, e.g., 'Relatório Técnico' e 'Tese de Mestrado' instead of 'Technical Report' and 'Master Thesis'. This file is used automatically if Portuguese is selected as the main language (see below);
% 		\item \textbf{defaults.tex} A file with the main default values for the package (institution name, faculty's logo, degree name and similars);
% 		\item \textbf{personaldataofthesis.tex} A file with the main default values for the package (identification of report as well as the author and juries);
% 		\item \textbf{template.tex} The main file. You should run  \texttt{LaTeX} in this one. Please refrain from changing the file content outside of the well defined area;
% 		\item \textbf{bibliography.bib} The bib file. An easy way to find to import citation into \texttt{bibtex} is select option \texttt{Show links to import citation into
% Bib\-Tex} in \href{http://scholar.google.pt/scholar_settings?hl=en&as_sdt=0,5}{\texttt{Scholar google settings}}.
% 		\item \textbf{thesisisel.cls} The  \texttt{LaTeX} class file for the thesis{} style. Currently, some of the defaults are stored here instead of \verb!defaults.tex!. This file should not be changed, unless you're ready to play with fire! :)
% 	\end{enumerate}
% \end{enumerate}

% Again, we would like to recall that all the user \texttt{LaTeX} files should be stored in the \verb!ThesisISEL! directory, and all the images in \verb!ThesisISEL/Chapters/img! directory.\todo[inline]{Yet another note!}
% section folder_structure (end)

% ===================
% = Package options =
% ===================
\section{State of the Art} % (fold)
\label{sec:state_of_the_art}

In this section, will be presented and explained the different frameworks that are commonly used in asynchronous data processing in several programming languages.

First of all, in the section 2.3.1, will be presented the \texttt{reactivex.io} project and how the \texttt{Observer Pattern} is used in this project to implement reactive streams with and without \textit{back-pressure} in several languages.
Secondly, in the section 2.3.2, will be presented the \textit{Microsoft´s .NET} \texttt{Async Enumerables} and how this approach diverges from approaches made in \textit{reactivex.io}, \textit{Javascript}, and in Kotlin with \textit{Kotlin Flow}
In the sections 2.3.3 and 2.3.4, will be respectively presented the Kotlin Flow  and the Javascript strategies for asynchronous data processing.

Finally, in the section 2.3.5, will be made an overview and taken conclusions about the different technologies presented, and how each approach can be used for different problems and objectives. Then, will be made a theoretical prediction on how each technology behaves in several know circumstances. 



%% section how_to_write_using_latex (end)
