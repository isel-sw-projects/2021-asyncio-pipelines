\chapter{Conclusions and Future Directions}

After the results obtained in the previous charpter, we can conclude that:

\subsection{.NET Environment Analysis}
In the .NET ecosystem, the examination of various strategies revealed nuanced performance distinctions. The parallel approach consistently outperformed its competitors, highlighting the power of parallel computing in efficiently processing tasks.

\subsection{Java/Kotlin Environment Insights}
Similarly, in the Java/Kotlin sphere, strategies leveraging parallelization emerged as clear winners. This underscores the importance of leveraging concurrent processing capabilities, especially in scenarios dealing with extensive datasets or complex computations.

\subsection{JavaScript Strategy Performance}
In JavaScript, the RxJS library showcased significant efficiency, especially in the "Grouping Words" task. This exemplifies the strength of RxJS in managing data streams efficiently, a critical consideration for web-based applications.

\subsection{Overarching Conclusions}
Across all environments, a consistent theme emerges is that the baseline approach, which doesn't use extensive pipeline frameworks, tends to perform better in these particular algorithms. 
However, there is a key detail that we must take in account, is that in many projects, which have several data sources and the complexity of building a solution from scratch is not an option, using these kind of frameworks helps a lot in deliver a solution with good code readability that enables easy code maintenance, which in enterprise enviroments thats a lote preferable than a solution from scratch scalability batter in its performance.


\section{Future Work and Final Thoughts}
In summarizing this thesis, it is evident that asynchronous IO technologies are not just a technical innovation but a catalyst for a broader transformation in the field of software engineering. These technologies have the potential to redefine how software is conceptualized, developed, and maintained. The journey undertaken in this thesis has shed light on the multifaceted nature of these technologies, revealing their complexities, challenges, and immense potential.

As the digital landscape continues to evolve, the importance of abstraction through the use of these tools that make easier to work around complex concepts, like huge data handling in real time. Its very important in the development, maintenance and debugging, making possible to have more robust and less fail prone systems.

In the years to come, as we delve deeper into the age of digital transformation, the role of asynchronous IO technologies in shaping the future of software engineering will become increasingly prominent. This thesis is but the beginning of an ongoing narrative in the realm of computing, one that promises to be as dynamic and evolving as the technologies it examines.

\end{document}
