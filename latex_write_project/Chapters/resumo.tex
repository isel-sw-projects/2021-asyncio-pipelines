\abstractPT  % Do NOT modify this line

Esta tese inicia com uma contextualização histórica até chegar ao estado da arte atual no que diz respeito às metodologias de coleta de dados programaticamente, com um foco particular na transição do paradigma de recolha e processamento de dados através de I/O, utilizando sistemas tradicionais ("single-thread"), para paradigmas mais modernos, como a "programação reativa".

Com esta contextualização, busca-se proporcionar ao leitor uma compreensão clara da evolução das práticas: do que era realizado no passado, por meio de abordagens de processamento linear e programação verbalmente densa e complexa, para o que é feito na atualidade, onde o nível de abstração permite ao programador desenvolver código menos prolixo, empregando técnicas de paralelismo e modelagem de dados eficientes, minimizando preocupações com detalhes de baixo nível anteriormente considerados.

Após apresentar o estado da arte e o contexto histórico, esta pesquisa detalha conceitos fundamentais, tecnologias emergentes e paradigmas de programação, levando a uma análise comparativa meticulosa entre diversas tecnologias atuais que oferecem soluções em Programação Reativa. Este segmento crucial examina tecnologias como RxJava e .NET AsyncEnumerables, investigando suas semânticas, soluções e implementações, com o objetivo de revelar as vantagens e desafios de cada uma no processamento e manejo de dados em grande escala e em tempo real.

A análise comparativa é cuidadosamente projetada para ressaltar a eficácia, escalabilidade e capacidade de resposta desses paradigmas programáticos, fornecendo uma avaliação detalhada de seu desempenho em contextos práticos. Ela visa esclarecer os fatores decisivos que influenciam a escolha de uma tecnologia em detrimento de outra, especialmente em situações que exigem a manipulação intensiva de dados.

Concluindo, esta tese apresenta uma avaliação minuciosa dos resultados, destacando as potencialidades e limitações das tecnologias examinadas. Os insights obtidos a partir do estudo comparativo são sintetizados para auxiliar na tomada de decisões informadas entre diferentes abordagens programáticas. Adicionalmente, são delineadas perspectivas futuras para a pesquisa, apontando para o desenvolvimento das tecnologias de coleta e processamento de dados.




\begin{keywords}
Palavras-chave (em português): Programação Reativa, Metodologias de Coleta de Dados, Paradigmas de Software, RxJava, .NET AsyncEnumerables, Análise Comparativa, Processamento de Dados em Tempo Real, Abstração de Programação, Paralelismo
\end{keywords}

\vspace{1em} % Isto adiciona uma linha extra em branco após a seção de palavras-chave.
