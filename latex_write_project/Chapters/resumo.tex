\abstractPT  % Do NOT modify this line

Esta tese examina a evolução das metodologias de recolha de dados programaticamente, focando-se na transição de sistemas tradicionais de recolha de dados através de I/O assíncrono para paradigmas modernos de programação reativa. Estes paradigmas representam uma mudança de processos bloqueantes e sequenciais para estratégias de manipulação de dados dinâmicas e responsivas, uma transformação impulsionada pelo aparecimento de novas técnicas e tecnologias de programação assíncrona.

Após apresentar uma visão histórica concisa para contextualizar o estudo, a pesquisa concentra-se numa análise comparativa entre tecnologias chave que permitem programação reativa, como o RxJava e os AsyncEnumerables da Microsoft, que utilizam streams reativos para aumentar a eficiência e escalabilidade na gestão de grandes volumes de dados. A metodologia de benchmarking é direta, baseando-se na recolha de tempos de operação em tarefas fundamentais — como, por exemplo, identificar a maior palavra num conjunto de dados e categorizar palavras pelo número de letras. Para tornar as métricas recolhidas mais representativas, foram utilizadas grandes quantidades de dados de forma a simular situações reais.

Esta abordagem pragmática permite uma avaliação direta das capacidades de processamento de dados em tempo real de cada tecnologia, sublinhando benefícios práticos e distinguindo os seus desempenhos.

Em suma, esta tese oferece uma análise comparativa entre várias tecnologias modernas de programação para a recolha de dados em grande escala, tendo também o objetivo de melhorar a compreensão deste tópico. Apresenta contexto histórico, conceitos-chave e as respostas tecnológicas atuais à leitura de dados em grande escala, contrastando com as práticas do passado.

\begin{keywords}
Palavras-chave (em português): Programação Reativa, Metodologias de Coleta de Dados, Paradigmas de Software, RxJava, .NET AsyncEnumerables, Análise Comparativa, Processamento de Dados em Tempo Real, Abstração de Programação, Paralelismo
\end{keywords}

\vspace{1em} % Isto adiciona uma linha extra em branco após a seção de palavras-chave.
