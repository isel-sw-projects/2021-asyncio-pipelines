The thesis{} style includes the following options, that must be included in the options list in the \verb!\documentclass[options]{thesisisel}! line at the top of the \texttt{template.tex} file.

The list below aggregates related options in a single item. For each list, the default value is prefixed with a *.

\subsection{Language Related Options} % (fold)
\label{sub:language_related_options}

You must choose the main language for the document. The available options are:

\begin{enumerate}
	\item \textbf{*pt} --- The text is written in Portuguese (with a small abstract in English).
	\item \textbf{en} --- The text is written in English (with a small abstract in Portuguese).
\end{enumerate}

The language option affects:
\begin{itemize}
	\item \textbf{The order of the summaries.} At first the abstract in the main language and then in the foreign language. This means that if your main language for the document in english, you will see first the abstract (in english) and then the 'resumo' (in portuguese). If you switch the main language for the document, it will also automatically switch the order of the summaries.
	\item \textbf{The names for document sectioning.} E.g., 'Chapter' vs.\ 'Capítulo', 'Table of Contents' vs.\ 'Índice', 'Figure' vs.\ 'Figura', etc.
	\item \textbf{The type of documents in the bibliography.} E.g., 'Technical Report' vs.\ 'Relatório Técnico', 'MSc Thesis' vs.\ 'Tese de Mestrado', etc.
\end{itemize} 

No mater which language you chose, you will always have the appropriate hyphenation rules according to the language at that point. You always get portuguese hyphenation rules in the 'Resumo', english hyphenation rules in the 'Abstract', and then the main language hyphenation rules for the rest of the document. If you need to force hyphenation write inside of \verb!\hyphenation{}! the hyphenated word, e.g. \\
\verb!\hyphenation{op-ti-cal net-works}!.
% section package_options (end)

\subsection{Class of Text} % (fold)
\label{sub:class_of_text}

You must choose the class of text for the document. The available options are:

\begin{enumerate}
	\item \textbf{bsc} --- BSc graduation report.
	\item \textbf{prepmsc} --- Preparation of MSc dissertation. This is a preliminary report graduate students at ISEL/IPL must prepare to conclude the first semester of the two-semesters MSc work. The files specified by 
	\begin{inparaenum}
	\item \verb!\dedicatoryfile! and 
	\item \verb!\acknowledgmentsfile! 
	\end{inparaenum}
	are ignored, even if present, for this class of document.
	\item \textbf{msc} --- MSc dissertation.
\end{enumerate}
%% subsection class_of_text (end)
%
%% ============
%% = Printing =
%% ============
\subsection{Printing} % (fold)
\label{sub:printing}

You must choose how your document will be printed. The available options are:

\begin{enumerate}
\item \textbf{oneside} --- Single side page printing, and
\item \textbf{*twoside} --- Double sided page printing.
\end{enumerate}

The article 50th, of Decree-Law No. 115/2013, requires the deposit of a digital copy of doctoral thesis and master's dissertations in a repository that is part of the RCAAP  repository\footnote{Repositórios Científicos de Acesso Aberto de Portugal}, \url{https://www.rcaap.pt}.  This deposit aims to preserve scientific work, as well as providing Open Access to scientific production is not restricted object or embargo.

For the reason explained above, we include the option to format your thesis in a way that presents well on screen and/or on paper.   But always remember that your work will be stored in the RCAAP portal in electronic format.
% subsection printing (end)

The available options are:

\begin{enumerate}
\item \textbf{onpaper} --- Format your thesis in a way that presents on paper or,
\item \textbf{*onscreen} --- on screen.
\end{enumerate}

% =============
% = Font Size =
% =============
\subsection{Font Size} % (fold)
\label{ssec:font_size}

You must select the encoding for your text. The available options are:
\begin{enumerate}
	\item \textbf{11pt} --- Eleven (11) points font size.
	\item \textbf{*12pt} --- Twelve (12) points font size. You should really stick to 12pt\ldots
\end{enumerate}
% subsection font_size (end)

% =================
% = Text encoding =
% =================
\subsection{Text Encoding} % (fold)
\label{ssec:text_encoding}

You must choose the font size for your document. The available options are:
\begin{enumerate}
	\item \textbf{latin1} --- Use Latin-1 (\href{http://en.wikipedia.org/wiki/ISO/IEC_8859-1}{ISO 8859-1}) encoding.  Most probably you should use this option if you use Windows;
	\item \textbf{utf8} --- Use \href{http://en.wikipedia.org/wiki/UTF-8}{UTF8} encoding.    Most probably you should use this option if you are not using Windows.
\end{enumerate}
% subsection font_size (end)

% ============
% = Examples =
% ============
\subsection{Examples} % (fold)
\label{ssec:examples}

Let's have a look at a couple of examples:

\begin{itemize}
	\item BSc graduation report, in portuguese, with 11pt size and to be printed one sided (I wonder why one would do this!)\\
	\verb!\documentclass[bsc,pt,11pt,oneside,latin1]{thesisisel}!
	\item Preparation of MSc thesis, in portuguese, with 12pt size and to be printed one sided (I wonder why one would do this!). Note that, \verb!pt! is declared by default, so it can be omitted: \\
	\verb!\documentclass[prepmsc,12pt,oneside,latin1]{thesisisel}!
	\item MSc dissertation, in english, with 12pt size and to be printed double sided on screen. Note that, \verb!twoside! and \verb!12pt! are declared by default, so it can be omitted: \\
	\verb!\documentclass[msc,en,utf8,onscreen]{thesisisel}!
\end{itemize}


The present document is defined according to the following settings:
\begin{Verbatim}[breaklines=true, breakanywhere=true]
\documentclass[msc,pt,twoside,12pt,a4paper,utf8,onscreen,hyperref=true,listof=totoc] {thesisisel}
\end{Verbatim}

% subsection examples (end)
	
\section{How to Write Using \texttt{LaTeX}} % (fold)
\label{sec:how_to_write_using_latex}

Please have a look at Chapter~\ref{cha:a_short_latex_tutorial_with_examples}, where you may find many examples of \href{http://tobi.oetiker.ch/lshort/lshort.pdf}{\texttt{LaTeX}} constructs, such as sectioning, inserting figures and tables, writing equations, theorems and algorithms, exhibit code listings, etc.

