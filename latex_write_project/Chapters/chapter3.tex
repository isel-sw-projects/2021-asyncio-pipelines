\chapter{ Implementations and results }
\label{cha:a_short_latex_tutorial_with_examples}

By far, in this document, was described how asynchronous IO data retrieval and proccessing is implemented in different techologies, and how different approaches try to provide the same result, which is: handling complex under the hood issues related with asynchronous IO but with readable code in format of operation pipelining with the maximun performance possible. 

One the things already saw, was that independently of the under the hood implementation in the different frameworks, the end result is code very similar in terms of semantics, for example, an asynchronous stream pipelining in Java and in C\#, are almost indentical although the data retrieving proccess being implemented with different approaches and proccesses. 

Taking this in account, in this charpter,the objective will be to show and describe several implementations made in the scope of this thesis and compare them in terms of performance. 

These implementations were made using different programming frameworks. The implementations consists in a program that asynchronously reads files and uses stream pipelining to proccess the data in real-time, enabling to benchmark the performance of the different tecnhologies performing the same task.

Objectively, in this charpter, will be presented several strategies that use different libraries and tecnhologies made in different languages, namely: \nameref{sec:dotnet_implementation}, \nameref{sec:java_implementation} and \nameref{sec:js_implementation}.

In the end, will be possible to compare the results and take conclusions, for example, the behaviour of the same strategy or tecnhology performance in different programming frameworks.


\section{.NET} % (fold)
\label{sec:dotnet_implementation}
As already explained, .NET framework provide a 


\section{JAVA/kotlin} % (fold)
\label{sec:java_implementation}

\section{javascript} % (fold)
\label{sec:js_implementation}

