\abstractEN % Do NOT modify this line

This thesis examines the evolution of data collection methodologies, transitioning from traditional asynchronous I/O systems to modern reactive programming paradigms such as Reactive Programming and Kotlin Flow. These paradigms represent a shift from static, sequential processes to dynamic, responsive data handling strategies, a transformation driven by recent advances in programming techniques.

After presenting a concise historical overview to contextualize the study, the research focuses on a comparative analysis of key technologies like RxJava and .NET AsyncEnumerables, which utilize reactive streams to enhance the efficiency and scalability of large data volume management. The benchmarking methodology used in this research is straightforward, based on time measure of similar operations across the different tecnologies used in our implementations. The tasks that were used were fundamental and simple, such as identifying the largest word in a dataset and categorizing words by letter count, in a huge data set to simulate a world-like scenario.

This pragmatic approach allows for a direct evaluation of each technology's real-time data processing capabilities, underlining practical benefits and distinguishing their performances. 

In conclusion, this work provides a focused comparative analysis to enhance understanding of data collection technologies, while providing key concepts and demonstrating the advancements from traditional to modern tools, making a contribution on how these technologies compare to each other and perform depending of the situation.

\begin{keywords}
    Keywords (in English): Reactive Programming, Data Collection Methodologies, Software Paradigms, RxJava, Rx.NET, Async Enumerables, Comparative Analysis, Real-time Data Processing, Programming Paradigms Evaluation, Technology Performance Comparison.
\end{keywords}