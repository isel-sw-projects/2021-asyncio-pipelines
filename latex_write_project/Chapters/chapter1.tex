% 
%  chapter1.tex
%  ThesisISEL
%  
%  Created by Matilde Pós-de-Mina Pato on 2012/10/09.
%
\chapter{Context and Progress status}
\label{cha:introduction}


This report has the objective of informing the reader, about the current status of the development of the dissertation named \textit{From asynchronous IO to reactive stream pipelines}, that is being written by the student \textit{Diogo Paulo de Oliveira Rodrigues}.
Firstly, to understand the current state of development, it is important to make clear what will be the objective and structure of the document that is being written.

This paper aims in the first place, to make an overview on what tools were historically available to deal with non-blocking IO operations and asynchronous programming tools, and what are the state-of-the-art solutions available today to deal with this problem.
After presenting state-of-the-art concepts and technologies, these are then compared in terms of performance through the development of practical test cases, written in several programming languages widely used today.

Taking in account what was explaining above, the dissertation will aim to have the following main structure: 
	1 - Introduction and motivation
	2-  Concepts and State of the art
	3- 	Test case development
	4- 	Results analysis and conclusions

Since this report is preliminary, this structure may have to be slightly changed in the course of the development.

